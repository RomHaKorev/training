\usejava


\title{Crafting with SOLID principles}
\subtitle{a step to be a craftsman}
\author{
  Olivier Albiez \texttt{<oalbiez@azae.net>} \\
  Yann Danot \texttt{<yann@arolla.fr>}
}
\date{}


\begin{document}


\frame{\titlepage}


\begin{frame}{SOLID principles}
  \begin{center}
    \Huge{ \textcolor{red}{S}.O.L.I.D. }

    \Large{ \textcolor{red}{Single Responsibility} Principle }
  \end{center}
\end{frame}


\begin{frame}{Single Responsibility - Definition}
  \twocolumns{
    \center \includegraphics[width=0.3\textwidth]{images/swiss-army-knife}
  }{
    \center \includegraphics[width=0.3\textwidth]{images/toolbox}
  }

  \center \large {An artefact should have one and only one reason to change,
  meaning that an artefact should have only one job.}
\end{frame}


\begin{frame}{Single Responsibility - Sample}
  \sample{1-s}
\end{frame}


\begin{frame}{Single Responsibility - Smells}
  \begin{itemize}
    \item Large Class
    \item Long Method
    \item Lot of methods
    \item High Coupling/Low cohesion
    \item Helper class
    \item Multiple functional/technical concepts at the same
  \end{itemize}
\end{frame}


\begin{frame}{SOLID principles}
  \begin{center}
    \Huge{ S.\textcolor{red}{O}.L.I.D. }

    \Large{ \textcolor{red}{Open / Close} Principle }
  \end{center}
\end{frame}


\begin{frame}{Open / Close - Definition}
  \center \includegraphics[width=0.3\textwidth]{images/television}

  \center \large {Objects or entities should be open for extension, but closed for modification.}
\end{frame}


\begin{frame}{Open / Close}
  \sample{2-o}
\end{frame}


\begin{frame}{Open / Close - Smells}
  \begin{itemize}
    \item Complex switch/Lot of ifs
    \item High cyclomatic complexity
  \end{itemize}
\end{frame}


\begin{frame}{SOLID principles}
  \begin{center}
    \Huge{ S.O.\textcolor{red}{L}.I.D. }

    \Large{ \textcolor{red}{Liskov Substitution} Principle }
  \end{center}
\end{frame}


\begin{frame}{Liskov Substitution - Definition}
  \center \large {Every subclass/derived class should be substitutable for their base/parent class.}
\end{frame}


\begin{frame}{Liskov Substitution}
  \sample{3-l}
\end{frame}


\begin{frame}{Liskov Substitution - Smells}
  \begin{itemize}
    \item You have to check for the type provided (e.g. instanceof)
  \end{itemize}
\end{frame}


\begin{frame}{SOLID principles}
  \begin{center}
    \Huge{ S.O.L.\textcolor{red}{I}.D. }

    \Large{ \textcolor{red}{Interface Segregation} Principle }
  \end{center}
\end{frame}


\begin{frame}{Interface Segregation - Definition}
  \center \large {A client should never be forced to implement an interface that it doesn’t use or clients shouldn’t be forced to depend on methods they do not use.}
\end{frame}


\begin{frame}{Interface Segregation}
  \sample{4-i}
\end{frame}


\begin{frame}{Interface Segregation - Smells}
  \begin{itemize}
    \item Fat interface/Class with lot of methods
    \item Interface has multiple responsibilities
    \item Difficulties to expose a subset of responsibilities
  \end{itemize}
\end{frame}


\begin{frame}{SOLID principles}
  \begin{center}
    \Huge{ S.O.L.I.\textcolor{red}{D}. }

    \Large{ \textcolor{red}{Dependency Inversion} Principle }
  \end{center}
\end{frame}


\begin{frame}{Dependency Inversion - Definition}
  \center \large {Entities must depend on abstractions not on concretions. It states that the high level module must not depend on the low level module, but they should depend on abstractions.}
\end{frame}


\begin{frame}{Dependency Inversion}
  \sample{5-d}
\end{frame}


\begin{frame}{Dependency Inversion - Smells}
  \begin{itemize}
    \item Dependencies between classes (vs interface)
    \item Monolithic architecture
    \item Abstraction depends on details/implementation
  \end{itemize}
\end{frame}


\end{document}
